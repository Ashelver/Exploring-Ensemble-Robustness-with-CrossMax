% ----------------- APPENDIX ------------------

\chapter*{Appendices} \addcontentsline{toc}{chapter}{Appendices} 
\renewcommand{\thesection}{\Alph{section}} 

\section{Pseudocode of the original CrossMax Ensemble}
\label{alg:crossmax}
\begin{algorithm}
\caption{CrossMax Ensemble (Pseudocode)~\cite{fort2024ensemble}}
\begin{algorithmic}
\Require $Z_1, Z_2, \dots, Z_M \in \mathbb{R}^{B \times C}$ (logits from $M$ models)
\State Stack logits into $Z \in \mathbb{R}^{B \times M \times C}$
\State Normalize $Z$ by subtracting max logit across classes per predictor
\State Normalize $Z$ by subtracting max logit across predictors per class
\State Compute $k$-th largest logit per class across models
\Ensure Class with highest $k$-th (e.g., $k=2$) largest logit
\end{algorithmic}
\end{algorithm}


\section{Bound on the Size of CrossMax Prediction Sets}
\label{appendix:crossmax-bound}

Consider the CrossMax aggregation with $M$ predictors and $C$ classes.  
Let the logits matrix be
\[
\mathbf{Z} \in \mathbb{R}^{M \times C},
\]
where the $i$-th row $\mathbf{z}_i$ corresponds to the logits from predictor $i$, and $z_{i,c}$ denotes the logit for class $c$.

\paragraph{Step 1: Removing predictor dominance.}  
For each predictor $i$, subtract its maximum logit over all classes:
\[
\tilde{\mathbf{z}}_{i,c} = \mathbf{z}_{i,c} - \max_{c'} \mathbf{z}_{i,c'}.
\]
After this normalization, each predictor’s logits contain exactly one zero (corresponding to the original maximum), while the remaining entries are strictly negative (ignoring ties). Thus, there are exactly $M$ zeros in the normalized logits matrix, one per predictor.

\paragraph{Step 2: Removing class dominance.}  
For each class $c$, subtract the maximum logit among all predictors:
\[
\hat{\mathbf{z}}_{i,c} = \tilde{\mathbf{z}}_{i,c} - \max_{j} \tilde{\mathbf{z}}_{j,c}.
\]
Since all entries $\tilde{z}_{i,c} \leq 0$, the maximum within a class is always non-positive. Two cases arise:
\begin{itemize}
    \item If all $\tilde{\mathbf{z}}_{i,c} < 0$, then the maximum is strictly negative. After subtraction, one predictor attains zero for class $c$, and the rest are negative.
    \item If there exists some predictor $i$ with $\tilde{\mathbf{z}}_{i,c}=0$, then $\max_j \tilde{\mathbf{z}}_{j,c}=0$, so the logits for that class remain unchanged.
\end{itemize}

\paragraph{Step 3: $k$-th max aggregation.}  
CrossMax selects the $k$-th largest value across predictors for each class:
\[
s_c = \operatorname{k\text{-}max}_{i} \ \hat{\mathbf{z}}_{i,c}.
\]
A class $c$ belongs to the prediction set if
\[
s_c = 0.
\]
This requires at least $k$ predictors to have $\hat{\mathbf{z}}_{i,c} = 0$ for that class.

\paragraph{Bound on prediction set size.}  
Since there are exactly $M$ zeros in total (one per predictor after Step 1), the maximal number of classes that can satisfy $s_c=0$ is bounded by
\[
\left\lfloor \frac{M}{k} \right\rfloor.
\]

\paragraph{Interpretation.}  
This shows that the size of the CrossMax prediction set is \textbf{bounded above by} $\lfloor M/k \rfloor$.  
In our experiments with $M=6$ predictors and $k=2$, the prediction set size can never exceed $3$.  
Hence, the CrossMax rule inherently prevents excessively large ambiguous outputs, limiting uncertainty by design.
\hfill $\square$
